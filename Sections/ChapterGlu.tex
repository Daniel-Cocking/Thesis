\documentclass[class=article, crop=false]{standalone}
\usepackage[utf8]{inputenc}
\usepackage{graphicx}
\renewcommand{\baselinestretch}{1.5} 
\usepackage{wrapfig}
\usepackage{subcaption}
\usepackage{caption}
\usepackage{float}
\usepackage{xcolor}
\usepackage{amsmath}
\usepackage{blindtext}
\usepackage{import}
\usepackage[backend=biber, style=numeric, citestyle=nature, doi=false, isbn=false, url=false, eprint=false]{biblatex} % Imports biblatex package, Comment out main doc
\addbibresource{refs.bib} %Import the bibliography file, Comment out main doc

\begin{document} 
\label{section:Glucose}

\section{Introduction}

Deuterium metabolic imaging (DMI) is a magnetic resonance spectroscopic imaging (MRSI) method that enables substrates and metabolic products that are labelled with the non-radioactive hydrogen isotope, deuterium ($^2$H), to be mapped in vivo. The scientific impact of $^2$H and usage in the real world has increased greatly since its existence was first theorised\cite{Urey1932AConcentration}, and it was not long before its potential for biological applications was recognised\cite{Schoenheimer1935DeuteriumMetabolism, Schoenheimer1938TheMetabolism}. Of particular interest is the use of glucose as the administered labelled precursor molecule, which can provide a direct probe of glucose metabolism. Whilst the most common current uses for $^2$H in vivo involves deuterated glucose, initially it was D$_2$O that was ingested to elevate $^2$H levels for use in nuclear magnetic spectroscopy (NMR)\cite{Brereton1986PreliminarySpectroscopy, Irving1987InSpectroscopy}, to look at lipid metabolism. The most useful forms of deuterated glucose are those whose carbon-bonded hydrogen atoms in the first and sixth positions have been replaced by deuterium because these substitutions are the only ones that are transferred to pyruvate, and then to either lactate (Lac) or to a combination of glutamate and glutamine (Glx) via the tricarboxylic acid (TCA) cycle in the mitochondria. For this reason, [6,6'-$^2$H$_2$] glucose has been a common choice of labelling (isotopologue), providing twice the number of deuterium labels as would [1-$^2$H] glucose, and has been used in many \textit{in vivo} studies from preclinical work\cite{Lu2017QuantitativeSpectroscopy, Meerwaldt2023InImaging} to the demonstration in humans\cite{DeFeyter2018DeuteriumVivo, Roig2022Deuterium7T}. Lactate and Glx as well as non-metabolised glucose and water (natural abundance plus an additional amount caused by label-loss during various processes) become detectable via either choice of labelling. The detected lactate and Glx provides information about the cell’s propensity to metabolise glucose via glycolysis or the oxidative phosphorylation pathway, and thereby an important clinical potential of DMI is identified since many tumour cells exhibit an increased tendency for glycolysis, which manifests in higher than usual lactate production;  the well-known Warburg effect\cite{Warburg1956OnCells}.    

This technique allows information on downstream metabolite concentrations to be quantified as well as tracked to obtain metabolic flux measurements, without the use of ionising radiation in a similar way to what is most commonly used positron emission tomography (PET) scans. Besides tumours and other disease states, brain metabolism may also be altered, albeit temporarily, as part of its normal functioning. This might occur when a part of the brain is activated by a task or stimulus, such as by visual stimulation. It has been previously shown that there is metabolic activation in the visual cortex following visual stimulation\cite{Kushner1988CerebralStimulation, Beland-Millar2018FluctuationsStimulation}. Consistently using $^1$H lactate increases\cite{Prichard1991LactateStimulation., Sappey-Marinier1992EffectSpectroscopy, Fernandes2020MeasurementT}, and glucose decreases\cite{Lin2012InvestigatingT} in the visual cortex following visual stimulation has been measured. and $^{13}$C\cite{Chhina2001MeasurementSpectroscopy} MRS, $^{31}$P has also been used to measure metabolism changes\cite{Sappey-Marinier1992EffectSpectroscopy} however glucose and lactate signals are not present here. It is important to note that significant increases in glutamate and decreases in glutamine have also been shown\cite{Lin2012InvestigatingT}, since Glx is the combination of the two metabolites a statistical change in Glx is unlikely. 
 
In its simplest implementation of spectroscopic imaging, DMI is relatively straightforward to set up with a pulse-acquire chemical shift imaging (CSI) sequence, usually without the need for water suppression because of the low concentration of naturally occurring deuterated water. In most cases, the DMI spectra are also simpler to analyse than the proton spectra. If using glucose as the labelled substrate, there are just four metabolites to consider: water (HDO), glucose, Glx, and lactate. This relative simplicity can be regarded as a positive attribute in the context of potential clinical translation. However, the generally low SNR of DMI is not a favourable characteristic, and means that data is often acquired with low spatial resolution often around 8 ml\cite{DeFeyter2021DeuteriumFuture, deGraaf2020OnImaging} in volume, with the smallest to date being reported in humans is 2.97 ml\cite{Ruhm2022Dynamic9.4T}, at a high magnetic field strength of 9.4 T. Low SNR and spatial resolution can potentially be improved upon by performing DMI scans through indirect detection using $^1$H MRSI\cite{vanZijl2020SpectroscopicFluxes, Bednarik2021DeuteriumBrain, Niess2023Reproducibility3T, Ruhm2022Dynamic9.4T}, with the additional benefit of not requiring deuterium-specific hardware, but at expense of often requiring a more complex acquisition scheme and post-processing analysis. Another method of increasing SNR in DMI is to implement de-noising during post-processing analysis, this has been shown to improve SNR in MRSI using low rank approximations\cite{Nguyen2013DenoisingApproximations}. Similar techniques using Tucker decomposition\cite{Tucker1966SomeAnalysis, Bader2007EfficientTensors} have been applied to DMI MRSI data with notable SNR improvements\cite{vonMorze2021ComparisonT, Kreis2020MeasuringMRI}.

More than just D$_2$-glucose has been used to investigate in vivo metabolism, [6,6'-$^2$H$_2$] fructose has been used to investigate liver cancer\cite{Zhang202366-2H2Cancer}. Fructose was found to have a similar spectral appearance as the glucose, with slightly different kinetics. After deuterated acetate ([$^2$H$_3$] acetate) is intravenously injected it has been used to investigate myocardial metabolism\cite{Wang2021NoninvasiveImaging} and tumour metabolism\cite{DeFeyter2018DeuteriumVivo}, where acetate accumulation and glx changes are tracked. One approach to increasing the deuterium signal is to use a form of deuterated glucose with a larger number of deuterium labels. Since there are twelve hydrogen atoms in the glucose molecule, all twelve could be substituted with deuterium atoms. However, those in the hydroxyl groups tend to exchange rapidly with the surrounding water and therefore are not usually useful in metabolic applications. However, the seven locations in which the deuterium atoms are directly carbon-bonded are less labile and potentially useful, yielding the form [1,2,3,4,5,6,6'-$^2$H$_7$]glucose (D$_7$-glucose). Compared with [6,6'-$^2$H$_2$]glucose (D$_2$-glucose), the deuterium spectrum would contain a factor of 7/2 more components, many of which would overlap due to the broad linewidths, producing a gain in SNR. The deuterium atoms in the C1 and C6 positions (in the absence of label-loss) will be transferred to lactate or Glx molecules\cite{DeFeyter2020DeuteriumBrain}, producing a gain of 3/2. Therefore, it is expected that the use of D$_7$-glucose will increase the SNR and reliability of detected signals for glucose, Glx, and lactate, compared to using D$_2$-glucose. In addition, the four remaining deuterium labels in the positions C2 – C5 of D$_7$-glucose are transferred directly or indirectly to water during glycolysis, and will therefore contribute to an increased HDO signal\cite{Mahar2020HDOMetabolism, Mahar2021DeuteratedGlucose}. The HDO (deuterated water) signal increase that is a result from metabolism has been shown to be directly proportional to the increase in downstream metabolites (glx and lactate)\cite{Mahar2021DeuteratedGlucose}, which implies that regular non-spectroscopic imaging of the HDO signal increase can be used as a measure of the Warburg effect with improved spatiotemporal resolution compared to MRSI/CSI techniques. 

The primary aim of this study is to measure the difference in vivo metabolite signal/concentration changes for HDO, glucose, Glx and lactate in the brains of healthy human participants following ingestion of D$_2$-glucose or D$_7$-glucose. CSI scanning, de-noising and a sophisticated and robust fitting routine is used to track the change in metabolite signals. Metabolite signals for each metabolite was found to be larger for all metabolites after ingestion of D$_7$-glucose, concentration values were also found to be larger in each metabolite except glucose. This shows that Warburg maps with better SNR can be obtained following ingestion of D$_7$-glucose that would imply a better CNR between healthy and diseased tissue in patients with brain tissues. Secondarily, the possibility of detecting differences in metabolite concentrations due to an applied visual stimulus was investigated.  

\section{Methodology}

\subsection{Particicpants}

%%%%%%%%%%% Fifteen healthy human participants were scanned using a 7T Philips Achieva Scanner and a two-channel $^2$H/$^1$H birdcage coil purchased from Rapid Biomedical. The study participation was split into two in-person visits to the MR centre (which were $>$ seven days apart), the first included a screening visit which took $<$one hour. During the screening the participants gave informed consent and were checked that they do not meet the exclusion criteria. There are many criteria for this study which all involved checking participants are overall considered 'healthy'. More specifically this consisted of a BMI of $>$25 kg/m$^2$ ($>$27 kg/m$^2$ for males if their waist circumference $>$94 cm), uncontrolled high blood pressure, any significant medical condition (including diabetes, asthma and epilepsy) and being between the ages of eighteen and sixty. Each participant also underwent a finger-prick blood-glucose test which had to be less than 7.8 mmol/L. 

Ethical approval was received from the Faculty of Medicine and Health Sciences Research Ethics Committee (ref. no. FMHS 306-0621) at the University of Nottingham to recruit 15 healthy participants for this study. Informed consent was received from all participants and recruited if they had a BMI $<$ 25 kg/m$^2$ (or less than 27 kg/m$^2$ for males if their waist circumference was $<$94 cm), had a normal heart rate and blood pressure, a blood glucose concentration of $<$7.8 mM (finger-prick test), were between 18 and 60 years old, and had no significant medical conditions or issues related to safety in the MR scanner. At the screening visit, participants were informed whether visual stimulation would be applied and that at least an eight hour fast is required on the day of scanning, this was checked using a second blood glucose level test (finger-prick) that had to be $<$5.6 mM. For those receiving D$_2$-glucose (n=8), 5 experienced a visual stimulus. For those receiving D$_7$-glucose (n=7), 4 experienced a visual stimulus.

\subsection{Scan Protocol}

Scanning for each participant was split into two parts, the baseline natural abundance scans before the glucose drink which are used for quantification and calibration lasting approximately 20 minutes, and the 90-minute scan session after the glucose drink was ingested. The drink was consumed in a maximum of eight minutes. Baseline measurements included a $^1$H scout scan for planning; a $^1$H MPRAGE scan (Tscan = 353 s, FOV = 224 x 224 x 140 mm$^3$, TR = 7.1 ms, TE = 2.6 ms, 1.4 mm isotropic voxels); a non-localised $^2$H spectrum (Averages = 16, Tscan = 17 s, TR = 1000 ms, TE = 1.1 ms, Bandwidth = 3000 Hz, Samples = 2048); a two-cm slice-selective $^2$H spectrum positioned over the lateral ventricles (Averages = 128, Tscan = 129 s, TR = 1000 ms, TE = 1.9 ms, Bandwidth = 3000 Hz, Samples = 2048) and a 3D $^2$H chemical shift image (CSI) covering the whole brain (Averages = 6, Tscan = 670 s, FOV = 180 x 180 x 120 mm$^3$, TR = 230 ms, TE = 2.4 ms, 15 mm isotropic voxels, Bandwidth = 1200 Hz). All the natural abundance scans were performed in the absence of visual stimulation. The participant was then brought out of the scanner and consumed the glucose drink, this contained 0.75g/kg (bodyweight) of either D$_2$ or D$_7$-glucose powder—purchased from CK Isotopes Ltd. (microbiological/pyrogen-tested product) and Merck Life Science UK Ltd. (endotoxin-tested product)—dissolved in 250 ml of water at room temperature. The participant was allowed to consume this in their own time, and when they indicated that they were ready for the second scanning session (<30 minutes later), were guided back into the scanner.

% Scan detail table instead of it included in text

In the second session, the two $^1$H scans were repeated, followed by five or six repeats of the three $^2$H scans. In the event that the participant needed to exit the scanner for a short period and re-enter, the 1H scans were repeated before continuing with the deuterium scans. If the participant was to be visually stimulated, the display would be activated during the CSI scans only and quiescent otherwise.  

Visual stimulation was achieved via an 8 Hz flashing, black and white, radial checkerboard, similar to what has been used previously\cite{Fernandes2020MeasurementT}. The visual display was projected onto a screen that the participants could observe while lying in the scanner by wearing prism glasses. Most of the participants who experienced visual stimulation (three that ingested D$_2$-glucose and four that ingested D$_7$-glucose) experienced a checkerboard flashing pattern that was active for 50 seconds followed by 10 seconds of a red cross on a grey background. However, two participants (both of whom ingested D$_2$-glucose) experienced a checkerboard flashing pattern that was active for 30 seconds followed by 30 seconds of a red cross on a grey background. Participants who received no visual stimulation were asked to close their eyes. In all cases, the scanner room lights were turned off. 

% \subsection{MR system and scan details}

% Scanning was performed on a 7T Achieva scanner (Philips Healthcare), operating at 45.8 MHz for 2H. A 26.4 cm inner-diameter, dual-tuned 1H/2H birdcage RF coil (Rapid Biomedical) was used for deuterium measurements and anatomical 1H images. 

% MPRAGE 1H scans were acquired with the following parameters: FOV = 224 x 224 x 140 mm3, 1.4 mm isotropic voxels, TR = 7.1 ms, TE = 2.6 ms, with a scan duration of 353 s. Non-localised 2H spectra were acquired using 16 averages, TR = 1000 ms, TE = 1.1 ms, flip angle = 90°, bandwidth = 3000 Hz, 2048 samples, with a scan duration of 17 s. Slice-selective 2H spectra were acquired from a 2-cm-thick axial slice positioned over the lateral ventricles, using 128 averages, TR = 1000 ms, TE = 1.9 ms, flip angle = 90°, bandwidth = 3000 Hz,  2048 samples, having a scan duration of 129 s. 3D 2H chemical shift images (CSI) covering the whole brain were acquired using 6 acquisition-weighted (Ref 59) averages, FOV = 180 x 180 x 120 mm3 , 15 mm isotropic voxels, TR = 230 ms, TE = 2.4 ms, flip angle = 62°, bandwidth = 1200 Hz, 256 samples, with a scan duration of 670 s.

The $^1$H MPRAGE image was converted to a NIFTI format using MRIcroGL, and then bias-field corrected using FSL-FAST\cite{Zhang2001SegmentationAlgorithm}. The corrected MPRAGE image was then brain extracted using FSL-BET\cite{Smith2002FastExtraction} with a robust centre estimation and to further remove any bias and any neck voxels, the fractional intensity threshold was allowed to vary between subjects along with the vertical gradient. The MNI-152 brain image with 2 mm isotropic voxels was registered to each image obtaining the affine matrix using FSL-FLIRT\cite{Jenkinson2001AImages, Jenkinson2002ImprovedImages} and twelve degrees of freedom. The MNI-152 brain image is non-linearly registered to the same image to obtain the warp-field using FSL-FNIRT\cite{AnderssonJ2008FNIRT-FMRIBsTool}, with the affine from linear registration used as an initial guess. The warp-field is then used to non-linearly register probabilistic maps from the MNI-152 space for the frontal and occipital lobes to the MPRAGE space. Finally, FSL-FAST\cite{Zhang2001SegmentationAlgorithm} is used on the brain-extracted MPRAGE image to probabilistic maps for cerebrospinal fluid (CSF), grey matter (GM) and white matter (WM), with the affine matrix used to improve initialisation. The CSF mask was manually segmented to only include the left and right ventricles. The maps are then binarised to obtain region of interest (ROI) masks. 

% \subsection{Image and spectral processing}

% The 1H MPRAGE image was converted to a NIfTI format using MRIcroGL (www.nitrc.org), and bias-field corrected using FSL’s FAST38, and then brain-extracted using FSL’s BET39 with a robust centre estimation and to further remove any bias and any neck voxels, the fractional intensity threshold was allowed to vary between subjects along with the vertical gradient. The MNI-152 brain image with 2 mm isotropic voxels (distributed with FSL (Ref 62)) was non-linearly registered to the corrected MPRAGE image to obtain the warp-field using FSL’s FNIRT42. The warp-field was then used to non-linearly register probabilistic maps from the MNI-152 space for the frontal and occipital lobes to the MPRAGE space. Finally, FSL’s FAST38 was used to segment the corrected MPRAGE image to form probabilistic maps for cerebrospinal fluid (CSF), grey matter (GM) and white matter (WM). The CSF mask was manually edited to only include the left and right ventricles. The maps were then binarised to obtain region of interest (ROI) masks. 

In some of the CSI spectra a noise spike was visible that would affect each voxel in the same frequency position, the spike only affected one data point. To correct this the data points immediately either side of the spike were averaged together, the spike was replaced with this value. Each CSI was denoised in the time-domain using a Tucker decomposition25 with a compression matrix size of [64 6 6 5] (spectral and 3 spatial dimensions), this is a similar compression ratio to what has previously been used with DMI26,27 and C$^{13}$ studies\cite{Brender2019DynamicHyperpolarization}. The FID’s are fit using an adapted version of the OXSA-AMARES MATLAB toolbox\cite{Vanhamme1997ImprovedKnowledge, Purvis2017OXSA:MATLAB}, which involves setting bounds, initial guess’ and prior knowledge for each of the metabolites. The chemical shifts of each metabolite from the fitting have a relative shift to the water peak\cite{Meerwaldt2023InImaging}. Usually when analysing data after D$_2$-glucose ingestion, the glucose peak is fit as a single peak at 3.8 ppm, since the chemical shift difference between anomers and deuterium labels are negligible. However, this is not the case for D$_7$-glucose\cite{Govindaraju2000ProtonMetabolites} therefore the anomeric and differing label contributions need to be accounted for, for consistency this has been implemented when analysing the data after D$_2$-glucose ingestion as well. All glucose peaks have been fit with the same amplitude; all peaks have the same phase other than the peak in position which has a different phase due to its large difference in chemical shift. The amplitudes of each position are converted to complex amplitudes and interpolated to the same resolution as the MPRAGE image. These maps are then averaged over the whole-brain, occipital lobe, frontal lobe, CSF, GM and WM then the complex amplitudes can be averaged over each ROI mask for each time point.

% Each CSI was denoised using a Tucker decomposition25 to reduce the original data matrix from [NFID, NAP, NRL, Nslice] = [256, 12, 12, 8], to a compression matrix size of [64, 6, 6, 4], similar to compression ratios used previously in DMI26,27 and 13C studies43. The FIDs were fitted using an adapted version of the OXSA-AMARES MATLAB toolbox44,45, which requires prior knowledge for each of the metabolites. The chemical shifts of glucose (both anomers), Glx, and lactate are assumed to be the same as those of the 1H chemical shifts (Ref 46) and were implemented in the fitting as relative shifts to the water peak11. Often in previous studies that have used D2-glucose, the spectrum has been fitted as a single peak at 3.8 ppm, since the chemical shift difference between the four resonance lines (two deuterium labels, two glucose anomers) are usually not discernible due to the relatively broad linewidths and low SNR. However, this is not the case for D7-glucose46 which has a larger number of spectral lines and range of chemical shifts. Therefore, the spectrum needs to be accounted for more accurately, taking into account the contribution from each deuterium label for both anomers and, for consistency, this approach has also been implemented when analysing the data after D2-glucose. The glucose peaks were fitted assuming a common scaling factor; all glucose peaks have the same phase other than the peaks from the C1 position which have a different phase due to large differences in chemical shift. The linewidths of peaks from the same anomer share the same value. For HDO, Glx, and lactate, only single components were assumed, with independent amplitudes, phases, and linewidths.

% The amplitude and phase of each metabolite peak at each voxel position were converted to complex amplitudes and interpolated to the same resolution as the MPRAGE image. These maps were then averaged over the whole-brain, occipital lobe, frontal lobe, CSF, GM, and WM ROIs (using the binarised segmentation maps) to obtain ROI-averaged amplitudes for each metabolite for each CSI, which provided amplitudes as a function of time relative to glucose ingestion.

These values were then either quantified into concentration values, normalised or corrected for their T1 variation and used as ratios between the two different types of glucose.

% \subsection{Concentration calculations}

% Concentrations C^(m )for each metabolite m were determined using the equation

% Equation 1 paper

% where A^m is the FID amplitude of the metabolite, N^m is the number of effective deuterium labels per metabolite molecule, E^m is the attenuation factor given by

% Equation 2 paper

% where T_1^m is the longitudinal relaxation time of the metabolite, T_Ris the repetition time, θ is the flip angle, and where k is a scaling constant. This constant, which is found to be ROI-dependent, is calculated by using the average water amplitude within a given ROI, by applying Equation 1 with an estimate of the natural abundance concentration calculated assuming an isotopic percentage for deuterium of 0.0156% (Ref 56), a concentration of pure water at 55.4 M, a factor of 2 because of the two hydrogen atoms in water, and an estimate of the percentage of water in the ROI. Cortical grey matter (GM) and white matter (WM) were assumed to be 84% and 69% water (Ref 57). The occipital and frontal ROIs were assumed to be comprised of 40% GM and 60% WM, resulting in a water content of 75%. The whole brain ROI was assumed to be 10% CSF, 36% GM, 54% WM, resulting in 77% water.

% Once k was calculated for each ROI, metabolite concentrations were calculated via Equation 1, with knowledge of the deuterium label numbers, N^m.

% The effective number of deuterium labels depends on whether D2-glucose or D7-glucose is ingested and, for Glx and lactate, label-loss. For D2-glucose, we have assumed the effective number of labels for water, glucose, Glx, and lactate is 1, 2, 1.2, and 1.7, respectively (Ref 52). For D7-glucose, we have assumed 1, 7, 0.9, and 1.3 (estimated from data acquired by Funk et al. (Ref 58) assuming glutamine and glutamate are present in approximately equal amounts). 

% Longitudinal relaxation times for glucose, Glx and lactate were assumed to be 67 ms, 139 ms, and 297 ms respectively (Ref 14), independent of ROI, number of deuterium labels, and of whether D2-glucose or D7-glucose was the metabolic precursor. For water (HDO), the T1 relaxation times were 510 ms, 320 ms, and 290 ms for CSF, grey matter, and white matter, respectively (Ref 6). These relaxation times were used to estimate relaxation times for each ROI based on their water percentage.


\printbibliography % Comment out main doc

\end{document}