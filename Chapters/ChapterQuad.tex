\chapter{Quadrupolar Effects using Heavy Water}
\label{Chap:Quad}

\section{Introduction}

$^2$H possesses an electric quadrupolar magnetic moment (Q$_{\text{deuteron}}=$ 0.286 fm$^2$)  because its nuclear spin (I=1) is larger than 1/2 \cite{Stone2015NuclearData}. This not only shortens the the relaxation times of $^2$H relative to $^1$H, but also introduces line splitting which is somewhat similar to the effect of J-coupling or dipolar interactions. The magnitude of separation in the formed doublet is dependent on the effect of ordering on the time-averaged direction of the local electric field gradient, with respect to the magnetic field that is experienced by the $^2$H nucleus \cite{Seelig1977DeuteriumMembranes, Eliav2016MultipleMRS}. When performing \ac{MRSI}, each voxel will contain information from multiple different tissue/water compartments which can therefore complicate spectral appearance due to the superposition of ordered (anisotropic) and disordered (isotropic) signals. The low available MR signal due to the low $^2$H \ac{NA} (0.015\%) means that quite often studies looking into quadrupolar effects are performed at high field  \cite{Gursan2022ResidualMuscle} and \cite{Ooms2015DoubleTissue}/or \cite{Damion2022DoubleLoading} high $^2$H abundances using \ac{D$_2$O} loading. Double quantum filtering (DQF) can be used to simplify the spectral behaviour by eliminating the signals from isotropic compartments \cite{Sharf1995DetectionNMR-Spectroscopy, Perea20072HDisc}. However this generally reduces the available \ac{SNR}. 

The combinations of spin operators ($I_x, I_y$ and $I_z$) can be represented as spherical operator tensors ($T_{l,p}$), where $l$ is the rank and $p$ is the coherence of the tensor, noting that $|p| \le l$. Since $^2$H has a spin $I=1$, no rank greater than two can be reached, therefore the only multi-quantum coherence level that can be achieved is the double quantum coherence represented by the $T_{2,\pm2}$ tensors, which only arise in an anisotropic medium. Use of \ac{DQF} suppresses the single quantum coherences ($T_{1,\pm1}$) leaving only signal from the $T_{2,\pm2}$ tensors in anisotropic regions. The use of \ac{DQF} to measure measure the anisotropy of tissues and fluids in the body such as intervertebral disc tissue  \cite{Ooms2015DoubleTissue}, brain water \cite{Assaf1997InSpectroscopy} and elastin \cite{Sun2010InvestigationNMR} can reveal vital information about tissue structure in health and disease e.g. degenerative disc disease \cite{Ooms2015DoubleTissue}. Sodium ($^{23}$Na) is an example of other spin $>$ 1/2 nuclei ($I=3/2$) where multiple-quantum filtered (MQF) scans have been implemented. MQF scans have been used to invesigate $^{23}$Na changes in the intracellular sodium environment, and such measurements have the potential to provide a marker of compromised ionic homeostasis in ischemia \cite{Tsang2012Triple-quantum-filteredT}.

\subsection{Aims}

In this work, healthy human participants ingested a \ac{D$_2$O} to increase their $^2$H abundance. The dependency of quadrupolar splitting frequency on angular orientation of skeletal muscle in a magnetic field was then measured, using \ac{CSI} data acquired from the forearm and the calf at 3T with an in-house built saddle coil and Helmholtz coil, respectively. Bulk and \ac{CSI} data were also obtained with \ac{DQF} and the effect of muscle orientation on \ac{DQF} signal was also explored.

\section{Theory}
\subsection{Quadrupolar Splitting}

The quadrupolar moment interacts with the local \ac{EFG} which can be represented as a combination of up to six electric potential tensor elements. These can be simplified to three principal axis elements ($V_{xx}, V_{yy}$ and $V_{zz}$). By definition the sum of these elements is 0, as the \ac{EFG} is a traceless tensor. $V_{zz}$ is defined as the largest element and is usually specified as the \ac{EFG} at the quadrupolar nucleus ($V_{zz} = e\cdot q$) \cite{Elliott2021WhatMedia}. The difference between $V_{xx}$ and $V_{yy}$ scaled by $V_{zz}$ is referred to as the asymmetry parameter ($\eta$).
\begin{equation}
    \eta = \frac{V_{xx}-V_{yy}}{V_{zz}}
\end{equation}
By considering the time-independent Hamiltonian ($H_Q$) of the quadrupolar interaction it is possible to find a mathematical generalised form.
\begin{equation}
    H_Q = \frac{eQ}{4I(2I-1)}[V_0(3I^2_z-\boldsymbol{I}^2) + V_{\pm1}(I_{\mp}I_z+I_zI_\mp)+V_{\pm2}I^2_\mp]
\end{equation}
$V_0$, $V_{\pm1}$ and $V_{\pm2}$ are the three principal axis elements combined to create new elements, which are complex, that represent the total \ac{EFG}. When considering the rotational transformation from the molecule's fixed reference frame to the laboratory fixed reference frame \cite{Seelig1977DeuteriumMembranes}, along with an assumed value of $\eta$ = 0 (uniaxiality), the total Hamiltonian simplifies for $^2$H ($I$ = 1)  \cite{Sharf1995DetectionNMR-Spectroscopy}. The form of this equation, now using its simplified principal axis elements is
\begin{equation}
    H = -g\beta_N\boldsymbol{I}\cdot\boldsymbol{H_0} + \frac{eQ(3\cos^2(\theta)-1)}{8}V_{zz}(3I_z^2-\boldsymbol{I}^2)
\end{equation}
where $g$ is the g-factor, $\beta_N$ is the nuclear magneton, $Q$ is the (scalar) quadrupole moment, $\boldsymbol{H_0}$ is the magnetic field, $e$ is the charge of an electron, $I$ is the nuclear spin (of deuterium) and $\theta$ is the angle between the electric field gradient and the magnetic field. The perturbed energy levels can be calculated using the simplified total Hamiltonian. There is an associated energy difference and therefore a frequency difference between states $m=1$ and 0 and 0 and 1 of.
\begin{equation}
    \nu_Q(\theta) = \frac{3}{2}\left(\frac{e^2qQ}{h}\right)\left(\frac{3\cos^2(\theta)-1}{2}\right)
    \label{eqn:Quad:Angle}
\end{equation}
where the \ac{RQC}, which gives the magnitude of the splitting, in this case is given in the first part of this equation.
\begin{equation}
    \omega_Q/2\pi = \frac{3}{2}\left(\frac{e^2qQ}{h}\right)
    \label{eqn:Quad:RQC}
\end{equation}

This derivation shows that as a result of the quadrupolar magnetic moment interacting with the \ac{EFG}, a splitting is observed that is caused by the perturbation of the energy levels. The frequency magnitude of this splitting effect is given by the \ac{RQC} and depends only on the orientation of the deuterated  molecules with respect to the applied magnetic field (Eq. \ref{eqn:Quad:Angle}). In an isotropic structure this splitting effect is not visible due to averaging resulting from molecular motion. In an anisotropic media such as the skeletal muscle fibres in the calf, the quadrupolar splitting is at a maximum and equal to the \ac{RQC} value when the muscle fibres are oriented parallel to the magnetic field ($\theta = 0^\circ$). The splitting effect can also be nulled ($\nu_Q = 0$) if the muscle fibres are oriented at the magic angle $\theta = 54.74^\circ$ ($\cos^2\theta=1/3$) to the magnetic field.

Therefore it is possible to measure the \ac{RQC} constant from these \ac{NMR} spectra which provides information on the anisotropy of the media beinginvestigated.

% This detected signal from the \ac{DQF} signal is related to a second rank tensor arising from the \ac{EFG}, which can only be formed in anisotropic media/phases. This technique is applied by manipulating the pulse sequence with alternating $\pi/2$ and $\pi$ RF pulses. 

\subsection{Quantum Filtering}

\begin{figure}
    \centering
    \includegraphics[width=1\textwidth]{Figures/Quad/DQF_Coherence.png}
    \caption{\textit{The coherence pathway and the RF pulse sequence used for \ac{DQF} acquisition. Noting that the tensor $T_{l,p}$ includes information on the rank ($l$) and on the coherence ($p$), which is indicated on the right side. The phase cycling of the last pulse ($\phi$) is x, y, -x, -y and the receiver phase ($\theta$) is x, -y, -x, y.}}
    \label{fig:Quad:Coherence}
\end{figure}

A pulse sequence that can be used to suppress single quantum coherences and measure signals using from double quantum coherences is shown in Fig. \ref{fig:Quad:Coherence} \cite{Sharf1995DetectionNMR-Spectroscopy}, where the pulse angles and the wait times are shown below. It is important that the correct phase cycling is used here to suppress unwanted coherences \cite{Bodenhausen1984SelectionExperiments}. The transmit phase cycle for the last pulse is $x,\:y,\:-x,\:-y$ and the receive phase is cycled as $x,\:-y,\:-x,\:y$. 

\begin{equation}
    \pi/2-\tau/2-\pi-\tau/2-\pi/2-t_1-\pi/2-t_2 \textrm{ (Acquisition)}
    \label{eqn:Quad:Pulse}
\end{equation}

This phase cycle suppresses the single quantum coherences and preserves the double quantum coherences meaning only the \ac{DQF} signal is measured. The obtained FID appears as two anti-phase Lorentzian lines separated by the \ac{RQC}. If this separation is small the presence of two resonances can be difficult to identify, as the Lorentzian-lineshapes will overlap. The amplitude of the \ac{DQF} FID follows a damped sinusoid of the form. 

\begin{equation}
    A\sin(2\pi\nu_q\tau)\exp(-2\tau/T_2)
    \label{eqn:Quad:Amplitude}
\end{equation}

\noindent Here $T_2$ is the transverse relaxation time, $A$ is the signal amplitude, $\nu_q$ is the splitting frequency and $\tau$ is known as the creation time and is the time between the first two $\pi/2$ pulses. An example \ac{DQF} spectra with corresponding fitting can be seen in Fig. \ref{fig:Quad:Ex_DQF}. The form of the signal described in Eq. \ref{eqn:Quad:Amplitude} assumes perfect application of the flip angles shown in Fig. \ref{fig:Quad:Coherence} and that the signal is on-resonance. The coherence transfer pathway with the changing tensor terms can be seen in Fig. \ref{fig:Quad:Coherence}.

\begin{figure}
    \centering
    \includegraphics[width=1\textwidth]{Figures/Quad/Example_DQF.png}
    \caption{\textit{\ac{DQF} spectrum from an axial 2 cm slice of the lower leg, acquired with $\tau$ = 6.5 ms. The plots show the fitting of the two independent Lorentzians to an anti-phase \ac{DQF} doublet. The imaginary part of the spectrum (absorption) is shown in (a), the real part (dispersion) is shown in (b). The fitting produces a value for the splitting of $\nu_q$ = 27.7 Hz, and mean $T_2^*$ = 11.3 ms. In each case the fits to the two signal components are also shown separately.}}
    \label{fig:Quad:Ex_DQF}
\end{figure}

\section{Scanning}

Measurements for this work were obtained in two separate investigations that both used the ingestion of \ac{D$_2$O} to increase the $^2$H concentration. The initial investigation was setup to look at $^2$H enrichment in skeletal muscle (see Chapter \ref{Chap:D2O}), however in initial experiments quadrupolar splitting was observed across the muscles of the calf (most notably in the tibialis anterior muscle). This motivated us to investigate the effect of quadrupolar splitting in skeletal muscle, quickly in the first investigation and more seriously in the second investigation. The first investigation was used to formulate the second study and improve the parameters chosen and to develop aims. Different loading routines were used in each investigation and both took place at different times. Due to the use of different loading routines the $^2$H abundances were different in different subjects and experiments, but the exact \ac{SNR} was not important for either investigations, as long as it was good enough to allow useful data to be obtained in a reasonable time frame. The loading routine for the first investigation was the same as was used in Chapter \ref{Chap:D2O}, and the loading routine used for the second investigation was the same as in Chapter \ref{Chap:Lipid}. Here only results from the second investigation are shown.
% Each part of the first investigation had a different number of participants, while the second investigation had three participants for each part (two of whom also took part in the previous investigation). 

All data was acquired using a Philips 3T Achieva scanner. All $^1$H anatomical scans were performed using the built-in body coil using a 3D \ac{GE} sequence. $^2$H data was obtained using in-house built coils details of which can be found in Section \ref{Chap:Theory:Coils} of Chapter \ref{Chap:Theory}. 

% \subsection{Study One}
% \subsubsection{Quadrupolar Splitting}
% \label{Chap:Quad:1:Split}

% After the initial \ac{D$_2$O} loading period was completed and the participants $^2$H enrichment had reached a steady state level. The left calves of two participants (A and B) were scanned using the in-house built $^2$H saddle coil. 3D \ac{CSI} were obtained with 10 mm isotropic voxels, NSA = 2, samples = 256, \ac{TE} = 6.2 ms, \ac{TR} = 500 ms, \ac{BW} = 750 Hz, FOV = 120x120x60 mm$^3$, the \ac{CSI} data was originally acquired to investigate the overall spread of $^2$H enrichment in skeletal muscle (see Chapter \ref{Chap:D2O}). However quadrupolar splitting was observed across the calf (most notably in the tibialis anterior muscle). This was used as motivation to investigate the effect of quadrupolar splitting in skeletal muscle.
% The participants were therefore scanned again using the in-house built Helmholtz coil designed to scan the forearm. The left forearms of A and B were scanned as this allowed us to easily bend the arm in the coil at different angles with respect to the $B_0$ field of the magnet, which allows us to change $\theta$ in Eq. \ref{eqn:Quad:Angle}. A $^2$H 3D \ac{CSI} with 10x10x15 mm$^3$ voxels, NSA = 2, 256 samples, \ac{TE} = 6.4 ms, \ac{TR} = 363 ms, \ac{BW} = 750 Hz and \ac{FOV} = 120x120x60 mm$^3$ is acquired for a range of different arm angles was acquired. During scanning, for each angle, participants were lay in a prone position with their left arm above their head. A 'straight' arm represented  0$^{\circ}$ angle respect to the $B_0$ field, participant A was then scanned with angles of 0, 10, 20, 30 and 40$^{\circ}$ in one visit and 50, 60, 70, 80 and 90$^{\circ}$ in a second visit. Participant B was scanned with angles 0, 85, 30, 15, 45, 60, 75 and -10$^{\circ}$ in a single visit. 
% The 3D \ac{CSI} spectra for the forearm was analysed using MATLAB and started with zero-order phase correction and masked based on the maximum value from the spectra of a specific voxel being larger than 20\% of the maximum value of all spectra. Each voxel within the mask was then fit to two complex Lorentzian curves with equal amplitudes, phases and linewidths where the frequency splitting between the two peaks is determined from fitting. The spectra is also fit to three complex Lorentzian curves where the two 'outer' peaks have equal equal amplitudes, phases and linewidths similar to before except now there is a third central peak with different amplitude, phase and linewidth to the other peaks the frequency splitting is now the frequency difference between the 'outer' two peaks only. The fitting is performed by minimising the sum of the squared difference between the proposed fit and the phase-corrected signal, this was done using in-built MATLAB function fmincon. The best fit is determined by comparing the final minimised value returned from fitting, whichever combination of Lorentzian curves gave the lowest value is kept and outputted for that voxel. The frequency splittings are then averaged over the masked region to give a single frequency splitting. This process was repeated for each angle, with the angle values being obtained from the angulation of the imaging/spectroscpy \ac{FOV}. All the splitting's were then fit to Eq. \ref{eqn:Quad:Angle}, where the total \ac{RQC} is determined. This was repeated for participants A and B.

% \subsubsection{Double Quantum Filtering}
% \label{Chap:Quad:1:DQF}

% $^2$H spectroscopy and \ac{CSI} were performed on lower legs and forearms in four healthy human participants (A,B,C and D). $^2$H spectra were acquired from a 2 cm axial slice of the calf or forearm, by using hard pulses in combination with \ac{OVS}. \ac{DQF} spectra were created by using the sequence in Fig. \ref{fig:Quad:Coherence}, which produces an anti-phase spectrum whereby the quadrupolar doublet peaks acquire a relative phase of 180$\circ$ (see Fig. \ref{fig:Quad:Ex_DQF}). This relative phase means that the signal vanishes for any components if their splitting frequency is small. \ac{DQF} spectra are phased to produce a symmetric line-shape (labelled as dispersion spectra in anti-phase in Fig. \ref{fig:Quad:Ex_DQF}).  
% \ac{DQF} spectra were acquired for various values of $\tau$, and \ac{TR} = 1000 ms, \ac{TE} = 0.58 ms, \ac{BW} = 3000 Hz, samples = 1024. The number of averages varied, depending on available signal, between 28 and 128. \ac{FID}s were processed using Matlab scripts and spectra were fit to a maximum of three independent Lorentzian line shapes.
% $^2$H 2D \ac{CSI} data were also acquired from a single 2 cm axial slice, again using \ac{OVS}. 2D \ac{CSI} data were acquired for the usual pulse-acquire sequence, and using the anti-phase \ac{DQF} sequence with $\tau$ = 6.5 ms. The voxel volume was 10x10 mm$^2$, \ac{TR} = 378 ms, \ac{TE} = 3.9 ms, samples = 256 and \ac{BW} = 750 Hz. For the \ac{DQF} \ac{CSI}, 16 averages were acquired, whereas only 4 were needed for the pulse-acquire \ac{CSI}.

% \subsection{Study Two}
% \subsubsection{Quadrupolar Splitting}

\subsection{Quadrupolar Splitting}

During the second main study, after the initial \ac{D$_2$O} loading period was completed and the participants $^2$H enrichment had reached a steady state level. $^2$H 3D \ac{CSI} data were acquired with 10 $\times$ 10 $\times$ 10 mm$^3$ voxels, \ac{FOV} = 120 $\times$ 120 $\times$ 50 mm$^3$, \ac{TR} = 500 ms, \ac{TE} = 6.2 ms, \ac{BW} = 750 Hz, samples = 256 and NSA = 2 in the forearm of three healthy human participants using an in-house built Helmholtz coil. Images were acquired in each subject with the forearm at 10 different angles $\sim$ ranging from 0$^\circ$ to 90$^\circ$ to the field. 
% The protocol here was similar to what followed in the above section (chapter \ref{Chap:Quad:1:Split}). 

In order to ensure that the forearm was always in the centre of the field, that a large enough range of angles were covered and that each participant was as comfortable as they could be. Each participant was removed from the scanner between acquisitions of data at different angles, which is why commonly scanning ran over two days. The protocol for each angle included  acquisition of a $^1$H scout scan and \ac{GE} anatomical images, two bulk spectra and finally a 3D \ac{CSI}. 
% For angles larger than 45$^\circ$ the data had to be acquired sagittally as oppose to transverse.

Whilst this study was similar to what was performed in the initial investigation, one of the major improvements made was in the analysis routine. It is difficult to estimate the angle of muscles of the forearm relative to the magnetic field. Here a paper printout of a compass attached to the top of the coil helped us to orient the arm in the scanner. In the initial work the angle used in the analysis was estimated from the $^1$H scout scan. Here the angle of the arm was calculated by measuring the angle of the ulna bone using the Mango software (downloaded from https://mangoviewer.com/download.html), which was found to be different from the angle estimated from the compass, as well as the angle used when planning the scanning.

Zero-order phase correction was applied to all spectra, as well as de-noising through using \ac{HOSVD} \cite{Bader2007EfficientTensors} with a compression core matrix of [64,6,6,3] (spectral, followed by three spatial dimensions). A binarised mask with the same spatial resolution as the \ac{CSI} data was then constructed, following thresholding of 35\% of the maximum spectral signal (the mask was then filled in using imfill). The OXSA-AMARES \cite{Purvis2017OXSA:MATLAB} toolbox in Matlab was used to fit three Lorentzian peaks to each masked voxel. These comprised of a central peak due to any signal from isotropic compartments plus a doublet due to quadrupolar splitting in any anisotropic compartments. Each peak in the doublet has the same linewidth and amplitude and all peaks are fit with the same phase. The initial estimate of the separation of the doublet varied depending on the angle of the arm in the scanner. The separation of the doublet was converted from ppm to Hz and then the values, along with the angle of the arm relative to the magnetic field, was fitted to Eq. \ref{eqn:Quad:Angle}.

% \subsubsection{Double Quantum Filtering}
\subsection{Double Quantum Filtering}

Bulk \ac{DQF} $^2$H non-localised spectra were acquired from a 2-cm axial slice of the forearm in three healthy human participants. Hard pulses were used in combination with \ac{OVS} for slice selection. Spectra were obtained via an anti-phase \ac{DQF} sequence \cite{Sharf1995DetectionNMR-Spectroscopy} whereby the peaks of the doublet acquire a relative phase of 180$^\circ$ to one another. \ac{DQF} spectra were acquired for a range of values of the creation time, $1\leq\tau\leq36$ ms with, \ac{TR} = 1000 ms, \ac{TE} = 0.58 ms, \ac{BW} = 3000 Hz, samples = 1024 and NSA = 56.

$^2$H 2D \ac{CSI} data were also acquired from single slices in both the lower leg and forearm, using \ac{OVS} for slice selection with the anti-phase \ac{DQF} sequence (\ac{DQF}-\ac{CSI}) with $\tau$ = 5 ms. Each voxel for the \ac{CSI} was 10 $\times$ 10 mm$^2$ in-plane, \ac{TR} = 1000 ms, \ac{TE} = 2 ms, samples = 256, bandwidth = 750 Hz, NSA = 8 with a slice thickness of 2 cm. $^1$H scout and 3D \ac{GE} (2 mm isotropic voxels, \ac{TR} = 20 ms, \ac{TE}, 2.1 ms, \ac{FOV} = 128 $\times$ 128 $\times$ 192 mm$^3$, NSA = 1) anatomical images were also obtained, along with $^2$H 2D \ac{SQF}-\ac{CSI} data with the same scan parameters as the $^2$H 2D \ac{DQF}-\ac{CSI} scans. The forearm measurements were then repeated with the arm at a range of angles (0$^\circ$, 30$^\circ$, 60$^\circ$ and 90$^\circ$) to the $B_0$ magnetic field. 

The non-localised spectra were apodised using a 5 Hz exponential line-broadening filter to increase \ac{SNR}, whilst the \ac{CSI} data was de-noised using a Tucker decomposition \cite{Bader2007EfficientTensors} with a core matrix of [32, 6, 6, 2] (time, two spatial and angular dimensions respectively) to increase \ac{SNR}. All \ac{DQF} spectra were fit using the OXSA-AMARES \cite{Purvis2017OXSA:MATLAB} toolbox in MATLAB, using two Lorentzian peaks with 180$^\circ$ phase difference, equal linewidths and equal amplitudes. Automatic zeroth-order phase correction was also applied to each spectra. Then signal amplitudes from four voxels from the volar compartment of the forearm were averaged across each subject and all angles and the spectra compared.

% \section{Results}
% \subsection{Study 1}
% \begin{figure}
%     \centering
%     \includegraphics[width=1\textwidth]{Figures/Quad/Calf_A.png}
%     \caption{\textit{Left: \ac{CSI} spectra (red) overlayed onto a $^1$H \ac{GE} image from participant A's left calf of a single slice, with fits of two Lorentzian peaks (green) and three Lorentzian peaks (blue). Right: Interpolated map of quadrupolar frequency splitting's overlayed onto the same $^1$H \ac{GE} image from participant A's left calf.}}
%     \label{fig:Quad:Calf_A}
% \end{figure}
% \begin{figure}
%     \centering
%     \includegraphics[width=1\textwidth]{Figures/Quad/Arm_A.png}
%     \caption{\textit{Left: \ac{CSI} spectra (red) overlayed onto a $^1$H \ac{GE} image from participant A's left arm at an angle of 0$^\circ$ to the $B_0$ field of a single slice, with fits of two Lorentzian peaks (green) and three Lorentzian peaks (blue). Right: Interpolated map of quadrupolar frequency splitting's overlayed onto the same $^1$H GE image from participant A's left arm.}}
%     \label{fig:Quad:Arm_A}
% \end{figure}
% Figs. \ref{fig:Quad:Calf_A} and \ref{fig:Quad:Arm_A} show spectra acquired from $^2$H \ac{CSI} data acquired in the lower leg and forearm (respectively). Though some of the spectra are analysed are analysed differently to others, being fit with two Lorentzian peaks (green) as opposed to three (blue). On the right side of each Fig. the overlayed frequency splittings in Hz, after it has been interpolated to the same resolution as the $^1$H anatomical image. There is obvious in-homogeneity in the frequency splitting in Fig. \ref{fig:Quad:Calf_A} with increased splitting covering the \ac{TA} muscle and slightly the Gastrocnemius muscle. The frequency splitting is more homogeneous in the arm with the only notable decrease being present over the radial and ulna bones.
% \begin{figure}
%     \centering
%     \includegraphics[width=1\textwidth]{Figures/Quad/Split_Angle_1.png}
%     \caption{\textit{Graph Showing the variation of averaged quadrupolar frequency splittings of the arm against the angle to the $B_0$ field they were positioned at, for partcipant A (blue) and B (red). Along with the fit (dotted black) of both participant's data to Eq. \ref{eqn:Quad:Angle}, the splitting magnitude of this fit is 38 $\pm$ 2 Hz.}}
%     \label{fig:Quad:Split_Angle_1}
% \end{figure}
% Both the lower leg and forearms in Figs. \ref{fig:Quad:Calf_A} and \ref{fig:Quad:Arm_A} are aligned to the external magnetic field $B_0$. The forearm is then rotated over a range of angles (-10$^\circ$ to 90$^\circ$) with the previous methodology repeated averaging over the full \ac{ROI} obtaining a single value for the splitting and fitting to Eq. \ref{eqn:Quad:Angle}, which can be seen as a black dotted line in Fig. \ref{fig:Quad:Split_Angle_1}. The angle used here is the angle of the imaging/scanning \ac{FOV}. From this fitting a magnitude value for the splitting value, 38 $\pm$ 2 Hz, is found. From Eq. \ref{eqn:Quad:Angle} a minimum is found at the magic angle 54.74$^\circ$, a general trend of the splitting decreasing as the angle of the \ac{FOV} is seen here.
% \begin{figure}
%     \centering
%     \includegraphics[width=1\textwidth]{Figures/Quad/SQFDQF_CSI_1.png}
%     \caption{\textit{$^2$H \ac{DQF} \ac{CSI} (left) and pulse-acquire \ac{CSI} (right) data obtained from a 2 cm axial slice of the lower leg of the same volunteer with 1cm in-plane resolution. In the centre are spectra from three selected voxels so that a direct comparison can be made of the \ac{DQF} (left) and pulse-acquire (right) spectra from different muscle regions.}}
%     \label{fig:Quad:SQFDQF_1}
% \end{figure}
% Fig. \ref{fig:Quad:SQFDQF_1} shows $^2$H \ac{CSI} spectra overlaid on $^1$H \ac{GE} images of the lower leg, for both the \ac{DQF} \ac{CSI} (left) and pulse-acquire \ac{CSI} data (right). Individual \ac{DQF} and \ac{CSI} spectra from three voxels in different muscle groups are also shown in more detail. The red voxel is located in the \ac{TA}, the blue is from the gastrocnemius muscle and the green is from the largest muscle the Soleus. The difference in \ac{SNR} between the two datasets is the key feature between the two scans.
% \begin{figure}
%     \centering
%     \includegraphics[width=1\textwidth]{Figures/Quad/Bulk_DQF_1.png}
%     \caption{\textit{Deuterium \ac{DQF} spectra acquired from a 2 cm slice of the lower leg (of different volunteers) and forearm with different values of the creation time $\tau$.}}
%     \label{fig:Quad:Bulk_DQF_1}
% \end{figure}
% Fig. \ref{fig:Quad:Bulk_DQF_1} shows examples of four, phase-corrected, \ac{DQF} spectral datasets (three lower leg, one forearm) acquired from 2 cm axial slices, acquired with a range of $\tau$ values. The correct phase correction when displaying \ac{DQF} plots as the point at which the amplitude becomes positive is important. The calf measurements appear to evolve much faster compared to the calf, with negative amplitudes even being recorded. Each spectrum was fitted to two independent Lorentzian peaks (as illustrated in Fig. \ref{fig:Quad:Ex_DQF}) from which the mean amplitude was plotted as a function of. Examples of these \ac{DQF} plots are shown in Fig. \ref{fig:Quad:BuildUp} (two forearms, two lower legs). 
% \begin{figure}
%     \centering
%     \includegraphics[width=1\textwidth]{Figures/Quad/BuildUp.png}
%     \caption{\textit{\ac{DQF} amplitudes obtained by fitting anti-phase line shapes to spectra obtained from the lower leg and forearm (of different volunteers) as a function of the creation time $\tau$. Continuous magenta lines show fits to the \ac{DQF} amplitude variation with $\tau$ of the form Eq. \ref{eqn:Quad:Amplitude}. The amplitude term varies across datasets because of the use of different RF coils and the different levels of \ac{D$_2$O} loading in different volunteers.}}
%     \label{fig:Quad:BuildUp}
% \end{figure}
% \subsection{Study Two}

\section{Results}

Figure \ref{fig:Quad:Calf_Arm_CSI} shows individual slices from 3D \ac{CSI} data acquired from the lower leg and forearm, with the limb approximately aligned with the $B_0$-direction. The fitting is performed on a voxel-wise basis and is the same over the whole \ac{ROI} and uses the OXSA-AMARES toolbox \cite{Purvis2017OXSA:MATLAB} implementing prior knowledge, and the data has been de-noised. The splitting map has not been interpolated here, and can be seen to the follow the general trend in the underlying anatomical image. Doublets can be observed in many voxels, with residual quadrupolar splittings of 20 to 40 Hz. An increase in the splitting in the \ac{TA} muscle can be seen here, with more homogeneity of splitting values present in the forearm.

% The results here are similar to Figs. \ref{fig:Quad:Calf_A} and \ref{fig:Quad:Arm_A}. The main differences are that the fitting routine here is the same over the whole \ac{ROI} now (which uses prior knowledge), and de-noising has been applied and the frequency splitting has not been interpolated. The fitting matches the underlying data better here compared to Figs. \ref{fig:Quad:Calf_A} and \ref{fig:Quad:Arm_A}. Doublets can be observed in many voxels, with residual quadrupolar splittings of 20 – 40 Hz. The same trend over muscle groups as in Figs. \ref{fig:Quad:Calf_A} and \ref{fig:Quad:Arm_A} with an increase in the \ac{TA} muscle and more homogeneity in the forearm.

\begin{figure}
    \centering
    \includegraphics[width=1\textwidth]{Figures/Quad/Calf_Arm_CSI.png}
    \caption{\textit{Example slices from 3D \ac{CSI} of the lower leg (upper panels) and forearm (lower panels), showing spectra (left) and maps of the magnitude of splitting (right). Fits are in blue, CSI data in red. In both cases the limb was approximately aligned with the field.}}
    \label{fig:Quad:Calf_Arm_CSI}
\end{figure}

Figure \ref{fig:Quad:Arm_CSI} shows how the \ac{CSI} spectra from the forearm change as the limb is oriented at different angles to $B_0$. It can be seen that as the forearm is angled close to the magic angle (54.74$^\circ$) the quadrupolar splitting vanishes. 

\begin{figure}
    \centering
    \includegraphics[width=1\textwidth]{Figures/Quad/Arm_CSI_Angle.png}
    \caption{\textit{\ac{CSI}data from slices of the forearm acquired at two different angles (9$^\circ$, approximately along the field (upper) and 59$^\circ$ (lower), close to the magic angle) to $B_0$ (left panels). Averaged spectra are shown for all angles in the right panel. Quadrupolar splitting is evident in the 9$^\circ$ \ac{CSI} data, but not seen in the 59$^\circ$ \ac{CSI} data. The average spectra are consistent with quadrupolar splitting that varies with $3\cos^2\theta - 1$ as shown in Eq. \ref{eqn:Quad:Angle}.}}
    \label{fig:Quad:Arm_CSI}
\end{figure}

 Figure \ref{fig:Quad:Split_Angle_2} plots the averaged quadrupolar splitting frequencies against angle of the ulna with respect to the applied field. A fit to the expected variation in Eq. \ref{eqn:Quad:Angle} provided an average value for the splitting amplitude across all voxels and participants of 32 $\pm$ 1 Hz. 
 % Which is similar to Figure \ref{fig:Quad:Split_Angle_1} as well as the methodology used to create the Figure except for the improved fitting as well as calculating the angle of the arm using the angle of the bone as oppose as the FOV box. The fit of the splitting frequencies follows the measured splittings better in this case compared to Figure \ref{fig:Quad:Split_Angle_1}.

\begin{figure}
    \centering
    \includegraphics[width=1\textwidth]{Figures/Quad/Split_Angle_2.png}
    \caption{\textit{Average quadrupolar splitting frequencies as a function of forearm angle to $B_0$, fitted to the form of Eq. \ref{eqn:Quad:Amplitude}. Data points are the average splitting over the forearm in each 3D \ac{CSI} data set, with data acquired at 10 different angles to the field from 3 subjects. Error bars show standard deviation of the splitting over the volume.}}
    \label{fig:Quad:Split_Angle_2}
\end{figure}

Non-localised \ac{DQF} spectra from the forearm of three participants are plotted as a function of creation time ($\tau$) in Fig. \ref{fig:Quad:Bulk_DQF_2}. The biggest differences between participants visible here is the presence of signal at larger $\tau$ values.
% Which is similar to Figure \ref{fig:Quad:Bulk_DQF_1} except more $\tau$ values are used here especially at shorter values. The new fitting method was used here and therefore led to more. The overall behaviour of the forearm spectra here is similar to the forearm spectra in Figure \ref{fig:Quad:Bulk_DQF_1}. accurate fits.

\begin{figure}
    \centering
    \includegraphics[width=0.8\textwidth]{Figures/Quad/Bulk_DQF_2.png}
    \caption{\textit{\ac{DQF} spectra obtained from a 2 cm axial slice across the forearm (aligned approximately with $B_0$), as a function of the anti-phase \ac{DQF} sequence creation time, $\tau$. Dispersion mode spectra are shown for three subjects, so that the anti-phase doublets produce a symmetric spectrum with maximum amplitude at the centre frequency. The average values of $\nu_q$ found from fitting to these doublets are 36 $\pm$ 7 Hz (top), 31 $\pm$ 5 Hz (middle), and 36 $\pm$ 5 Hz (bottom).}}
    \label{fig:Quad:Bulk_DQF_2}
\end{figure}

Figure \ref{fig:Quad:SQFDQF_2} shows a comparison of 2D pulse-acquire \ac{CSI} and \ac{DQF}-\ac{CSI} data along with fits for the lower leg, highlighting spectra in individual voxels in the \ac{TA} and soleus muscles. 
% This Figure is similar to Fig. \ref{fig:Quad:SQFDQF_1} except here the \ac{SNR} is slightly better here thanks to de-noising as well as showing consistent fits to each voxel.

\begin{figure}
    \centering
    \includegraphics[width=1\textwidth]{Figures/Quad/SQFDQF_CSI_2.png}
    \caption{\textit{Slices of the lower leg from a \ac{CSI} sequence (left) and a \ac{DQF}-\ac{CSI} ($\tau$: 5 ms) sequence (right), highlighting selected spectra from two voxels: \ac{TA} (orange), soleus (green).}}
    \label{fig:Quad:SQFDQF_2}
\end{figure}

2D \ac{DQF} \ac{CSI} spectra for the lower leg at two different angles, (a) 11$^\circ$ and (b) 65$^\circ$, along with fits are shown in Fig. \ref{fig:Quad:Arm_DQF}. The \ac{SNR} in data acquired at the angle closer to the magic angle is notably smaller compared to the lower angle \ac{CSI}, however fitting was still accurate even in the lower angle data. Also shown are the four voxel \ac{ROI}s that are used for comparison in Fig. \ref{fig:Quad:DQF_CSI_Angle}.

\begin{figure}
    \centering
    \includegraphics[width=1\textwidth]{Figures/Quad/Arm_DQF.png}
    \caption{\textit{$^2$H 2D \ac{DQF} \ac{CSI} obtained from the same subject's forearm using a Helmholtz coil at two different angles to the field, (a) 11$^\circ$ and (b) 65$^\circ$. The yellow boxes show the \ac{ROI} used to compare spectra in Fig. \ref{fig:Quad:DQF_CSI_Angle}}.}
    \label{fig:Quad:Arm_DQF}
\end{figure}

In Fig. \ref{fig:Quad:DQF_CSI_Angle} a change in \ac{DQF} amplitude is obvious across each subject and across all angles with minimums being in the angle closest to the magic angle which is consistent with Eq. \ref{eqn:Quad:Amplitude}. The \ac{SNR} is notably lower in subject three's spectra compared to subjects one and two, however it still exhibits the same behaviour with fitting still being possible.

\begin{figure}
    \centering
    \includegraphics[width=1\textwidth]{Figures/Quad/DQF_CSI_Angle.png}
    \caption{\textit{Four $^2$H spectra averaged together for four angles across three subject's forearms, obtained from \ac{DQF}-\ac{CSI} spectra stacked on top of each other. Angles relative to the $B_0$ field were measured from the orientation of the ulna bone.}}
    \label{fig:Quad:DQF_CSI_Angle}
\end{figure}

\section{Discussion}

\subsection{Quadrupolar Splitting}

The residual quadrupolar splitting of the \ac{HDO} spectrum, seen in Fig. \ref{fig:Quad:Calf_Arm_CSI}, is evidence of local ordering of the tissue, which been previously observed in muscle, tendon, cartilage, and nerves \cite{Gursan2022ResidualMuscle,Sharf1995DetectionNMR-Spectroscopy,Perea20072HDisc,Eliav2016MultipleMRS}. All measurements of angular dependencies in this work were made on the forearm due to ease of scanning at multiple specific angles. Other work has performed similar scanning on the lower leg and found magnitude of splittings for different muscle groups. That work was based on \ac{NA} $^2$H signals, but used higher field strength (7T) and measurements at just two different angles (0$^\circ$ and 45$^\circ$) \cite{Gursan2022ResidualMuscle}. The splitting map observed in the calf in Fig. \ref{fig:Quad:Calf_Arm_CSI} is similar to what has been found previously at $\sim$ 0$^\circ$. Because of the inhomogeneous spatial splitting in the lower leg, separate \ac{ROI}'s for each muscle group had to be used when comparing splitting values. This is not necessary in the forearm (as can be seen in Fig. \ref{fig:Quad:Calf_Arm_CSI}) which is why only a single  ROI is used here. In the calf the largest splitting arises in the \ac{TA} muscle group, probably because the fibres of this muscle align closely \cite{Gursan2022ResidualMuscle} with the $B_0$-direction with leg along the field. However this could also indicate a more ordered environment in which the water resides in the \ac{TA}. Separate \ac{ROI} analysis in the main three forearm muscle groups (mobile wad, dorsal and volar compartments) would have been interesting to see if any small difference was present. However, due to the large voxels and low \ac{SNR} from using the built-in body coil for the anatomical images, it was difficult to identify the different muscle groups from the $^1$H images.

% In the first investigation MATLAB's fmincon function was used to determine the minimum of the sum of the residuals squared between two/three Lorentzian peaks and the experimental data. The number of peaks that provided the lower minimum value was used for the fitting. In a purely anisotropic medium where the entire signal has been separated into a doublet through quadrupolar splittings a two-peak Lorentzian would accurately model the data. 

Due to the low spatial resolution of the \ac{CSI} scans realistically there is a combination of separated doublets and single peaks present which is represented by a three-peak fit. A three-peak fit could be used to fit all voxels as long as each peak is fully resolved, this is often not the case here as the linewidth is often larger than the separation. Therefore in some cases a three-peak fit is not ideal, which is why the condition of lowest sum of squared residuals is used to choose the apropriate fit. The OXSA-AMARES MATLAB toolbox has already been shown to improve fitting by use of prior knowledge. This has led to increased reliability in fitting here which allows the use of more consistent three-peak fitting in the time domain. 

% The change in fitting routine has not changed the overall spatial trend in the maps of the separation frequencies, which can be seen in Fig. \ref{fig:Quad:Calf_Arm_CSI}.

Improving the fitting, de-noising  or the change in angle determination (\ac{FOV} box angle or the angle of the ulna) has increased the fitting accuracy to Eq. \ref{eqn:Quad:Angle} in Fig. \ref{fig:Quad:Split_Angle_2}. It can not be categorically stated which is most responsible for the increased accuracy of fitting. However, due to the fact the angle of the ulna can be significantly different to the angle of the \ac{FOV} box, it is suspected that this is the driving force for the increase in fitting accuracy.

\subsection{DQF}

As is seen with the \ac{DQF} filtered \ac{CSI} data it can be difficult to identify where the regions of maximum \ac{DQF} signal are present due to the lack of discrimination of the anti-phase peaks. 

However, in Fig. \ref{fig:Quad:SQFDQF_2} an increase in the signal in the \ac{TA} and gastrocnemius msucle and a decrease in the soleus muscle is visible which is consistent with separation maps from pulse-acquire \ac{SQF} results in Fig. \ref{fig:Quad:Calf_Arm_CSI}. This confirms that the strength of \ac{DQF} signal depends on the degree of tissue ordering, as the magnitude of the quadrupolar splitting has already been shown to depend on tissue ordering in Eq. \ref{eqn:Quad:Angle}. 

% Therefore, analysing a creation-time ($\tau$) non-localised \ac{DQF} series is more complicated as it could contain multiple splitting frequencies ($\nu_q$) due to the ordering and T$_2$ times which appear in Eq. \ref{eqn:Quad:Amplitude}. 

% This is why the behaviour between the forearm and lower leg spectra and amplitude variations differ so much in Fig. \ref{fig:Quad:BuildUp}. And whilst fittings are plotted in Fig. \ref{fig:Quad:BuildUp} it is clear this model does not suit the data very well, this could be due to resonance offsets, flip angle errors and potentially a range of quadrupolar frequencies \cite{Sharf1995DetectionNMR-Spectroscopy} (as has already been pointed out). 

% Fits to eight \ac{DQF} datasets from study one (which included the datasets in Fig. \ref{fig:Quad:Bulk_DQF_1}) produced a mean and standard deviation of $\nu_q$ = 39.7 $\pm$ 11.5 Hz. However, in all curves, a global maximum is seen at approximately 2$\tau \approx$ 15 ms, which would correspond to approximately $\nu_q \approx$ 33 Hz. As a comparison, mean values obtained by fitting all \ac{DQF} spectra in each dataset produced $\nu_q$ = 29.5 $\pm$ 3.7 Hz, which is in closer agreement with values similarly measured in the lower leg \cite{Gursan2022ResidualMuscle}. 

% Whilst still having the similar characteristics to the forearm curves in Fig. \ref{fig:Quad:BuildUp}, the repeated forearm curves in Fig. \ref{fig:Quad:Bulk_DQF_2} have a slower evolution and are more similar to the calf curves.

Increasing the number of $\tau$ values in investigation two to what is used in Fig. \ref{fig:Quad:Bulk_DQF_2} in a similar range to what has been used previously, a more complete evolution of the \ac{DQF} signal in the forearm is possible. By using OXSA-AMARES \cite{Purvis2017OXSA:MATLAB} more reliable and accurate fitting is possible even at larger $\tau$ values with lower SNR. From having a more complete model extra features are visible in the amplitude changes in Fig. \ref{fig:Quad:Bulk_DQF_2}, such as extra signal that is not an exponential decay, as shown in Eq. \ref{eqn:Quad:Amplitude}. 

By using the forearm as opposed to the calf, the partial voluming of multiple splittings is minimised. The $\tau$ evolution in Fig. \ref{fig:Quad:Bulk_DQF_2} has not been modelled in detail as it is clear that a more complicated model is needed, involving off resonance and flip angle effects. Difficulties in fitting this data arise from poor flip angles due to \ac{RF} inhomogeneity, off resonance effects and contributions from multiple splitting frequencies \cite{Sharf1995DetectionNMR-Spectroscopy}. 

Whilst the signal evolution with $\tau$ could not be correctly modelled, the angular response of \ac{DQF} signal has been explored in Fig. \ref{fig:Quad:DQF_CSI_Angle}. A full response curve similar to Fig. \ref{fig:Quad:Split_Angle_2} was not possible as only four angles have been measured. However, it is enough to see that the data approximately follows Eq. \ref{eqn:Quad:Angle} with minimum signal around the magic angle (54.74$^\circ$) and maximum with a straight arm (0$^\circ$ to the $B_0$ field).

\subsection{Future and Limitations}

These are the first results using $^2$H for \textit{in vivo} \ac{DQF} spectroscopy in human subjects, and as such there are areas for improvement and developing the technique further. By using dual-tuned coils it would be possible to acquire high resolution anatomical $^1$H images as well as $^2$H data, which would allow improved segmentation and \ac{ROI} analysis. In this case it would allow separation of specific muscle groups and \ac{DQF} signals to be identified. An improved \ac{RF} coil with multiple channels for the $^2$H would also increase \ac{SNR}, which would allow for better spatial resolution, and reduce effects from multiple quadrupolar separations.

On the topic of \ac{RF} coil improvements, if the $B_0$ field strength is increased say from 3T to 7T this would decrease the individual peak linewidths. Whilst the splitting frequency does not vary with field strength, minimised linewidths means the peaks will be better resolved which can lead to improved fitting. The \ac{SNR} will also improve with field strength, and therefore scan time can potentially be reduced, which could lead to improved dynamic scanning.

Whilst the \ac{HDO} \ac{DQF} peaks overlap which creates one strong central signal, a \ac{DQF} imaging sequence could be used instead of a spectroscopic sequence. This would reduce scan time which could allow for more averaging to improve \ac{SNR}, improve spatial resolution on \ac{CSI} acquisitions or acquire more dynamic data ie. with more angles or more $\tau$ values.

\section{Conclusion}

Here the first \textit{in vivo} $^2$H \ac{DQF} datasets from human subjects have been reported. These will potentially allow crucial information on the ordering of local tissue to be obtained. Experiments have been performed at a clinical field strength of 3T which shows that in the future (with more improvements to study protocol). This technique could be used to investigate effects of diseases that have so far only been shown on \textit{ex vivo} data \cite{Ooms2015DoubleTissue, Sharf1995DetectionNMR-Spectroscopy, Perea20072HDisc, Sun2010InvestigationNMR}. The quadrupolar separation has also been quantified in different muscle groups in the lower leg and in the calf as a whole, as well as the angular dependence on this behaviour. By improving the \ac{RF} coil used, increasing the field strength and using imaging sequences this technique shows potential for uncovering useful information about tissue ordering.

% \printbibliography % Comment out main doc

% \end{document}