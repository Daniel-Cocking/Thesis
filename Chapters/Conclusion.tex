\chapter{Conclusion}
\label{section:Conclusion}

In this thesis \ac{MRI} and \ac{MRSI} tuned to the $^2$H resonance at 3T and 7T has been used to explore some important $^2$H MR parameters such as T$_1$ relaxation times. The metabolic behaviour of $^2$H glucose has also been investigated \textit{in vivo} using \ac{DMI} at 7T. Due to this work not being performed on human scanners \textit{in vivo} at the \ac{SPMIC} previously, some of the hardware used, such as \ac{RF} coils, where in-house built and the details of choosing the type of coil and the building of the coil have been described here as well. $^2$H MRSI with $^2$H glucose is regularly used to investigate metabolic diseases such as brain cancer due to the increased \ac{CNR} from metabolite concentration maps. The aim of this work is to successfully lay the ground for future studies to investigate brain tumours in patients \textit{in vivo} using \ac{DMI} at 7T. At the time of writing ethics have been approved for patient studies. 

\section{Chapter Overviews}

To obtain a complete understanding of this relatively new technique (\ac{DMI}) it is important to establish an initial understanding of the background and history of the technique. This is done in Chapter \ref{Chap:Introduction}. The basic biology underlying metabolism and the pathways that $^2$H in a labelled compound that is ingested will take through the body can be found here. The historic use of $^2$H in MR research is described in this chapter from its discovery, to its use in heavy water in the 1980's in animal models, to its current most popular use in the form of $^2$H labelled glucose.

The physics that underpins the development of the experimental work in this Thesis is outlined in Chapter \ref{Chap:Theory}. Initially this describes how \ac{NMR} data is obtained going from a microscopic to a macroscopic picture. Most of \ac{MRI} and \ac{MRS} research is performed by tuning to the universe's most abundant element $^1$H, therefore the differences when tuning to a quadrupolar nuclei such as $^2$H are explained here. \Ac{MRSI} techniques are used in numerously throughout this thesis so the acquisition of this type of data is explained and the options of different scans that can be used to obtain this data are explained. The mathematical approach to analysing $^2$H spectra is shown in this chapter as well as the strategies to improve the intrinsically low \ac{SNR}. Finally the theory behind building \ac{RF} coils is described here along with details on the electrical components.

To accurately design and optimise scanning protocols it is important to know the physical properties of the compound being scanned. One of the most important of these is the T$_1$ relaxation time, which is a key factor in selecting the \ac{TR} and the flip angle used in measurements. The T$_1$ times for \textit{in vivo} deuterated water in \ac{CSF}, \ac{GM} and \ac{WM} are described in this chapter to increase the accuracy in later concentration quantification and improve future protocol design. Participants that were undertaking a study into cell proteomics ingested heavy water, giving rise to a hundred fold increase in the $^2$H concentration. \ac{MEGE} images with a range of \ac{TR}'s were acquired allowing joint analysis of signal variation with \ac{TE} and \ac{TR} to provide values of T$_2^*$ and T$_1$. Two of the participants were also scanned when they first began ingesting heavy water at regular intervals so that the time course of the increase in $^2$H levels is explored.

Before any work on patients can be undertaken it is important that initial measurements that involve healthy human participants are performed. This is so that scanning protocols can be optimised for comfortability and reduce the needed scan time. Such measurements are described in Chapter \ref{Chap:Glucose}. Healthy participants were given D$_2$- and D$_7$ glucose and concentration maps were obtained for \ac{HDO}, glucose, Glx and lactate. The change of signal and concentration with time is shown, and it is found there is considerable gain in \ac{SNR} when using D$_7$-glucose versus D$_2$-glucose for each metabolite, which would potentially result in increased \ac{CNR} for healthy tissue against diseased tissue.

Investigation of lipid metabolism can provide useful insights in studying metabolic diseases including diabetes, currently one of the most popular methods obtain data to investigate this involves performing invasive biopsies following heavy water ingestion. Chapter \ref{Chap:Lipid} reports the first results using $^2$H tuned MRSI to investigate increased lipid signals following ingestion of D$_2$O. Increases in the lipid $^2$H signal from two out of three participants in the abdomen and in 1 out of 3 in the calf. T$_1$ relaxation times were also found at natural abundance of \ac{HDO} and deuterated lipid in skeletal muscle were also found in this chapter.

One of the different characteristics of $^2$H compared to $^1$H is its quadrupolar moment which can cause spectral broadening/splitting in ordered tissue such as the muscle. The magnitude of quadrupolar splitting in both the forearm and the calf was quantified in different muscle groups (where possible) in Chapter \ref{Chap:Quad} along with the relationship between the \ac{DQF} signal and splitting magnitude. The evolution of \ac{DQF} with varying creation time ($\tau$). This was made possible due to the increased \ac{SNR} obtained from participants ingesting D$_2$O. This is the first time this has been attempted using $^2$H resonance in humans \textit{in vivo}. 

\section{Future Directions}

A common tool throughout the whole thesis is the use of some element of de-noising. Whilst apodisation/line-broadening has already been shown to be useful at increasing \ac{SNR} it is problematic as quite commonly spectral lineshapes can overlap, it can also negatively impact quantification. To overcome this \ac{HOSVD} is used throughout this thesis with matrices up to 4-dimensions and has shown to be pivotal in improving fitting accuracy through increased \ac{SNR}. There is a current trend in the field currently whereby this strategy is implemented with the level of den-noising used as similar to other works etc. It would be great to see more of a focus on \ac{HOSVD} as a de-noising tool (as oppose to just a data reduction tool) and how far this method can be pushed without sacrificing spatial/temporal/spectral information. One of the ways in which spectral fitting was improved was through the addition of prior knowledge, it would be great to see the same thought process applied to de-noising for example does data/matrix shape effect de-noising?

It is important to note that whilst this thesis shows how to obtain and analyse \textit{in vivo} $^2$H data, this work only sets the foundation. As has been pointed out there are current issues with how metabolic information is obtained clinically, using \ac{PET}. $^2$H has the potential to provide similar information. However, this first has to be shown in patient studies as the usefulness of this technique can not truly be proven until direct comparisons are made with current clinical work. Whilst $^2$H glucose and Glx are important metabolites in the investigation of brain tumours, arguably one of the most important metabolites is lactate. Currently it is difficult to optimise a scan for imaging lactate as signal levels do not reach far above natural abundance in healthy participants. Brain tumours cause a significant increase in the lactate produced which can be detected using $^2$H. Also, patients potentially will not be able to stay in the scanner as long so the total scan time has to be reduced, which means it can be difficult to obtain as much useful information. Therefore, it would be great to see this technique not only used in brain tumour patients but also in other metabolic diseases such as \ac{AD} or \ac{PD}.

One of the difficulties when trying to track $^2$H incorporation into lipids after D$_2$O ingestion is the overlap of signals. When only one peak in a spectrum is visible this means imaging can directly be used to obtain signal values of that specific signal. Therefore if the increased \ac{HDO} signal was able to be nulled only the lipid would be visible, therefore if the same sequence was used to acquire image data a greater \ac{SNR} could potentially be obtained. However, for this to be possible almost near perfect water suppression is needed which can be difficult to achieve. In the \ac{DQF} spectra the data suffers from low \ac{SNR} as well and only has one signal present, therefore this sequence could be made into an imaging sequence which could potentially result in smaller voxels which would hopefully reduce effects from partial voluming creating multiple separations in the same voxel. Therefore in the future it would be great to see more applications of $^2$H imaging as oppose to spectroscopy for more sparse spectra situations It's important to note this wouldn't be possible for \ac{DMI} experiments due to the multiple metabolite peaks being present.

\section{Closing Remarks}

In this thesis it has been shown not only possible to obtain \textit{in vivo} $^2$H spectroscopic information in a reasonable time frame, in humans, as well as $^2$H \ac{MRI} and \ac{MRSI} at 7T. Whilst the current most popular use for $^2$H in MR research is in \ac{DMI}, this thesis has shown the wide potential of uses for $^2$H and why this nucleus is becoming more popular. I hope this thesis helps push clinical strategies away from either ionising and/or invasive procedures and helps open up a pathway to more $^2$H MR research and into a place where $^2$H capabilities comes as standard in all clinical scanners.